\documentclass{article}
\usepackage[utf8]{inputenc}
\usepackage[czech]{babel}
\usepackage{url}
\usepackage{biblatex}
\usepackage{graphicx}
\usepackage{csquotes}
\addbibresource{references.bib}

\title{Specifikace softwarového díla \\
\&\\
Časový plán implementace \\
pro \\
\textbf{Generování jednoduchých nakreslení grafů}}

\author{Filip Čermák}

\begin{document}
\maketitle

\begin{center}
    Verze 1. 
\end{center}

{\centering
\section*{Anotace}}
Cílem projektu je generovat jednoduchá nakreslení úplných a případně obecných grafů, na kterých poté půjde ověřovat platnost známých kombinatorických domněnek.
Jako hlavní motivaci máme počítání průsečíkových čísel úplných grafů.

\clearpage
\tableofcontents

\clearpage

\section*{Tabulka revizí}
\begin{tabular}{ |p{3cm}|p{2cm}|p{4cm}|p{3cm}|  }
\hline
Jméno & Datum & Důvod změny & Verze\\
\hline
\hline
Filip Čermák & 30.4.2020 & Finální verze specifikace & 1.0\\
\hline
\end{tabular}

\clearpage

\section{Základní informace}

\subsection{Popis a zaměření softwarového díla}

{\it Jednoduchá} nakreslení grafů jsou ta, kde se žádné dvě hrany nekříží více než jednou, přičemž průnik ve vrcholu se počítá také jako křížení.

Program poslouží ke generování jednoduchých nakreslení úplných, případně obecných grafů. Na těchto vygenerovaných nakresleních půjde poté ověřovat platnost známých domněnek, například o průsečíkových číslech. 

Dále {\it průsečíkové} číslo grafu je minimální počet křížení křivek reprezentujících hrany přes všechna nakreslení daného grafu.

Jednou z motivací je zkoumání struktury jednoduchých nakreslení grafů pomocí tzv. $k$-$hran$, což by mohlo dát nový náhled například u Harraryho--Hillovy domněnky o průsečíkových číslech úplných grafů.


\subsection{Použité technologie}

\begin{itemize}
    \item Programovací jazyk C++, std/c++17
    \item Programovací jazyk C\#, Visual Studio 2019 Community
    \item C\# WindowsForms (knihovna tříd pro grafické rozhraní)
    \item Graph\# (knihovna pro vykreslování grafů, \url{https://archive.codeplex.com/?p=graphsharp})
\end{itemize}

\subsection{Odkazy(Reference)}

\nocite{monotone}
\nocite{mastersthesis}  
\printbibliography[heading=none]

\section{Stručný popis softwarového díla}

\subsection{Důvod vzniku softwarového díla a jeho základní části a cíle řešení}

Cílem projektu je pokusit se vyvrátit či případně na malých případech ověřit známé domněnky o jednoduchých nakresleních grafů.

V současnosti je nám známá pouze jedna databáze jednoduchých nakreslení od J{\" u}rgena Pammera~\cite{mastersthesis}. Tato databáze ovšem není volně dostupná a obsahuje pouze jednoduchá nakreslení úplných grafů. K vytvoření databáze jednoduchých nakreslení úplných grafů použijeme podobné metody a pokusíme se je rozšířit také na obecné třídy grafů.

Problém rozhodnout, jaký je minimální počet křížení v nakreslení daného grafu, je obecně velice obtížný.
I pro speciální třídy grafů se jedná o jedny z nejznámějších a nejstarších kombinatorických problémů o grafových nakresleních.
Například problém určit minimální počet průsečíků v nakresleních
úpl\-ných bipartitních grafů $K_{m,n}$ je starý přes 70 let a je známý jako \emph{Zarankiewiczova domněnka}~\cite{history}.
Jako první se těmito problémy zabýval matematik Paul Turán, kterého podobné otázky napadly, když byl v roce 1944 nucen pracovat v továrně na cihly.
Turán zde musel převážet cihly z několika továren do skladů pomocí vozíků, které pracovníci přesouvali po kolejnicích.
Největším problémem při převozu byla křížení kolejnic, protože na nich často cihly z vozíků popadaly.
Turána napadlo, že rozvržení kolejnic minimalizující počty těchto křížení by vedlo k úspoře času a práce.
Nebylo ovšem jasné, jak by takové rozvržení mělo vypadat a pro obecný počet $n$ továren a $m$ skladů se dodnes jedná o nevyřešený problém.
Podobná přes sto let stará úloha zvaná \emph{Tři domy a tři studně} je vlastně problémem určení minimálního počtu průsečíků v nakresleních grafu $K_{3, 3}$.
Zde lze ovšem poměrně snadno nahlédnout, že jedno křížení je nutné a zároveň dostačující.

Určení minimálního počtu průsečíků v nakresleních úplného grafu $K_n$ na $n$ vrcholech je také těžkým problémem známým pod mnoha jmény~\cite{history}, například jako \emph{Hararyho--Hillova domněnka}.
Tento velice známý a dosud nevyřešený pro\-blém pochází z 50. let, kdy jej poprvé formuloval umělec Anthony Hill.
Na tomto problému během posledních desítek let pracovala spousta známých matematiků a přesto, že existuje vzorec pro domnělý minimální počet průsečíků v nakreslení $K_n$, 
nepodařilo se dodnes dokázat, že neexistují lepší nakreslení.
Narozdíl od Zarankiewiczovy domněnky se však v posledních několika letech podařilo u Hararyho--Hillovy domněnky dosáhnout významného pokroku.

Cílem tohoto projektu by bylo s využitím vytvořené databáze jednoduchých nakreslení hlouběji prozkoumat metody, za pomoci kterých bylo těchto pokroků dosaženo, a případně dále
prohloubit naše znalosti o nakresleních minimalizujících počty průsečíků úplných ale i obecných grafů.

\subsection{Hlavní funkce}

Program se bude skládat ze dvou částí. Z programu pro generování jednoduchých nakreslení (\textbf{generátor}) a z programu (\textbf{vizualizér}) pro vizualizaci nakreslení vystoupených generátorem.

Kód generátoru bude psaný v jazyce C++ a bude implementovat datovou strukturu \textbf{Half-edge}. 
Kód vizualizéru bude psaný v jazyce C\#. Vizualizér poté vykreslí nakreslení vystoupené generátorem přehledně pro uživatele, který na něm bude moci ověřovat nějakou z domněnek o průsečíkových číslech, či jinou z domněnek o nakresleních grafů.

\subsection{Motivační příklad užití}

Příkladem problému, k jehož řešení by naše databáze mohla přispět, je slavná Harraryho--Hillova domněnka, která říká, že průsečíkové číslo úplného grafu $K_n$ je přesně $Z\left( n\right) :=\frac{1}{4}\left\lfloor \frac{n}{2}\right\rfloor \left\lfloor \frac{n-1}{2}\right\rfloor \left\lfloor \frac{n-2}{2}\right\rfloor \left\lfloor \frac{n-3}{2}\right\rfloor$. Jsou již známá nakreslení grafu $K_n$ se $Z(n)$ průsečíky, ale ukázat, že vždy existuje alespoň $Z(n)$ průsečíků je otevřeným problémem téměř 70 let.

V článku~\cite{monotone} je uvedeno zobecnění Harrahyho--Hillovy doměnky, které využívá $k$-$hran$.
Mějme $D$ jednoduché nakreslení grafu $K_n$. Poté se podívejme na orientovaný oblouk $\gamma$ reprezentující hranu $\{u, v\}$ pro různé vrcholy $u$ a $v$. Pro libovolný jiný vrchol $w$ řekneme, že je nalevo (napravo) od $\gamma$, pokud topologický trojúhelník $uvw$ je orientovaný proti (po) směru hodinových ručiček.
Hrana $\{u, v\}$ je {\it k-hranou}, pokud má právě $k$ vrcholů napravo nebo nalevo.

Zadefinujme $E_k(D)$ jako počet $k$ hran v daném nakreslení $D$.
Položme \[E_{\leq k}(D) = \sum_{i = 0}^{k}E_i.\] Obdobně můžeme zadefinovat \[E_{\leq\leq k}(D) = \sum_{i = 0}^{k}E_{\leq i}(D) = \sum_{i = 0}^{k}(k-i+1)E_i(D)\] a zcela analogicky \[E_{\leq\leq\leq k}(D) = \sum_{i = 0}^{k}E_{\leq\leq i}(D) = \sum_{i = 0}^{k}(k-i+1)E_{\leq i}(D) = \sum_{i = 0}^{k}{k+2-i \choose 2}E_i(D).\]

Zobecněná Harryho--Hillova domněnka~\cite{monotone} říká, že pro každé jednoduché nakreslení $D$ grafu $K_n$ a každé $k$, kde $0 \leq k < \frac{n}{2} - 1$ platí $E_{\leq\leq\leq k}(D) \geq {k+4 \choose 4}$.  

%Známý důkaz je pro domněnku o počtu $\leq\leq k$ hran, ale zatím jen pro specifické druhy nakreslení (úsečkové, monotonní a jiné). Tato domněnka říká, že v úplném grafu je $E_{\leq\leq k}\geq { k + 3 \choose 3}$.

Existuje verze této domněnky pro obecné grafy, která říká, že pokud má graf alespoň ${2k+3 \choose 2}$ hran, pak je $E_{\leq\leq\leq k}(D) \geq {k+4 \choose 4}$.
Tato domněnka je motivací pro generování jednoduchých nakreslení obecných grafů. 

\subsection{Prostředí aplikace}

Aplikace bude primárně cílená na Windows, ve kterém bude laděna. Program generátoru by měl fungovat i na operačním systému Linux.  

\section{Vnější rozhraní}

\subsection{Uživatelské rozhraní, vstupy a výstupy}

Oba programy budou lokální. 
Generátorem vystoupená jednoduchá nakreslení budou uložená do souboru, ve kterých budou dostupná pro program vizualizér.  

Typická interakce mezi uživatelem a vizualizérem bude možnost vytvoření si svého nakreslení grafu, zobrazení konkretního nakreslení vystoupeného generátorem a provádění jeho úprav (přidání/odebírání vrcholů, hran). Následně si bude uživatel moci zobrazit různé informace o daném nakreslení, které budou záviset na tom, co chce zrovna zkoumat, například $E_k(D), E_{\leq k}(D), E_{\leq\leq k}(D)$ nebo $E_{\leq\leq\leq k}(D)$, či počty průsečíků. 

\subsection{Rozhraní se software}

Databáze jednoduchých nakreslení vystoupených generátorem bude poté volně přístupná na \url{http://bitbucket.org/}, pokud by si chtěl někdo otestovat svou oblíbenou domněnku.

Nakreslení generátoru budou uložena ve formě fingerprintu (otisku), což je úsporný způsob reprezentace jednoduchého nakreslení grafu, použitý v \cite{mastersthesis}.
Pro graf $G$ a vrchol $v \in V(G)$ buď $\pi_v$ cyklickou permutací hran sousedících s tímto vrcholem. {\it Rotačním systémem} je potom množina $\pi = \{\pi_v : v \in V(G)\}$ těchto cyklických permutací, nebo též {\it rotací}.
{\it Fingerprint (otisk)} je kanonický rotační systém, kde vybereme lexikograficky minimální posloupnost rotací kolem vrcholů. Označme jako $\rho(v)$ lexikograficky minimální rotaci okolo vrcholu $v$. Potom otiskem je $\rho = (\rho_{min}(2), \dots, \rho_{min}(n))$ ($\rho_{min}(1)$ BÚNO můžeme vynechat, protože v otisku platí $\rho_{min}(1) = (2, 3, \ldots, n)$). 

\section{Detailní popis funkcionality}

\subsection{Generování jednoduchých nakreslení $K_n$}

Program generování probíhá na struktuře \textbf{Half-edge}. Pro graf $K_n$ prvně vytvoříme \uv{hvězdu}, která představuje středový vrchol $1$ spojený se všemi ostatní\-mi. Poté se budeme pokoušet vytvořit nové hrany, které ještě zbývají.
Tyto vrcholy nebudou vrcholy v běžném slova smyslu. Budou uchovány jako kružnice hran (tedy budou tvořit \uv{jednostrannou} stěnu). Každá taková kružnice bude obsahovat $n-1$ vrcholů, které budou uspořádány dle rotace daného vrcholu. Rotační systém totiž již kombinatoricky určuje jednoduché nakreslení.

Budeme si uchovávat pole $blocked[a][b][c][d]$, které nám bude říkat, jestli se hrany $\{a, b\}$ a $\{c, d\}$ mohou protnout, přičemž hrany, které mají společný vrchol, již považujeme za protnuté, aby bylo nakreslení jednoduché. Dále budeme mít seznam segmentů $segments[s]$, což jsou již nakreslené části hran, a také pole $indices[a][b]$, které si pro hranu $\{a, b\}$ bude pamatovat z jakého segmentu vede. Poté už bude jednoduché zjistit, mezi kterými vrcholy na daných kružnicích hranu právě vytváříme.

Vždy, když zjistíme, že danou hranu vytvořit za daných podmínek neumíme, vynoříme se z rekurze a zkusíme další možnost.

\subsection{Vizualizace jednoduchých nakreslení}

Poté bychom chtěli tato nakreslení vizualizovat, abychom mohli vidět jeho strukturu. To bude zajišťovat naše aplikace vizualizéru v jazyce C\#. Uživatel si bude moci přidat vrcholy. Nechat je spojit hranou, odebrat vrchol či hranu. 
Přidaní vrcholů znamená kliknutí na již stávající vrchol a protáhnutí hrany až do nového bodu v rovině, kde vytvoří se vytvoří bod (kvůli zachování souvislosti). Přidání hrany bude v podstatě analogické, až na to, že skončíme již v existujícím vrcholu. Odebrání hrany znamená přepočítání jejích sousedů, což je také triviální operace. Obdobně odebrání vrcholu znamená odebrání všech jeho hran a samotného vrcholu.
Samotné odebraní bude fungovat pomocí mazacího modu. Pokud jej budete mít zapnutý, pak stačí kliknout na hranu či vrchol a pokud se neporuší souvislost, bude daný element odebrán.

Dále bude možné si nechat postupně vizualizovat další a další jednoduchá nakreslení $K_n$ a zkoumat jejich vlastnosti. 
Odebrat z nich nějaký vrchol, hranu atd. jako v předchozím případě. 

\subsection{Vizualizace průsečíků a $k$-hran}

V grafické části si bude moci uživatel zvolit, zdali chce vědět počty průsečíků, ${\leq k}$-hran atd. 
Průsečíky pro nás budou zvláštními vrcholy, proto vystoupit jejich počet bude snadné. Na určení počtu $E_k(D)$ bude třeba projít celý graf a spočítat počty orientovaných trojúhelníků. V závislosti na tom, jestli půjde o obecný graf budeme muset počítat buď oba směry, nebo jen jeden. Další hodnoty už jsou pouhými sumacemi $E_k(D)$, jak jsme si je zadefinovali. 
Také bude možnost vidět tyto hodnoty přes sousedy jednoho vrcholů.
Poslední důležitá možnost bude při odebraní vrcholu či hrany zobrazí všechny invariantní hrany.

V aplikaci bude tedy 5 hlavních módů na vizualizaci vlastností, kde každá se bude volit tlačítkem. Vystoupení $k$-hran v barevném spektru bude vidět vždy.
První bude zobrazení průsečíků. Ty se zobrazí jako červené body v grafu.
Druha až čtvrtá možnost bude zobrazení všech ${\leq k}$-hran, ${\leq\leq k}$-hran, ${\leq\leq\leq k}$-hran, pomocí čárkovaní hran. Počet hran vrcholů grafu bude omezen 20, proto postačí barevné spektrum 10 barev.
Barvy skutečných hodnot hran zůstanou nedotčené, jenom uvidíme pohromadě jejich společné vlastnosti.
Pátá možnost bude zobrazení invariantních hran vůči odebraní vrcholu či hrany.

Hrana je {\it invariantní} $k$-hranou v nakreslení $D$ vůči vrcholu $v$ či hraně $e$, pokud zůstane $k$-hranou i v nakreslení, které vznikne z $D$ odebraním vrcholu $v$ či hrany $e$.

\clearpage

\section{Obrazovky}

\subsection{Vizualizér}

Na obrázku je příklad nakreslení grafu $K_5$ ve vizualizéru. Barevně jsou odlišené $k$-hrany dle palety dole. Po pravé straně jsou tlačítka jednotlivých módů vizualizéru. Dostupnými módy jsou \textbf{Přidávání}, \textbf{Mazání}, \textbf{Průsečíky}, \textbf{Počítání ${\leq k}$-hran}, \textbf{Počítání ${\leq\leq k}$-hran}, \textbf{Počítání ${\leq\leq\leq k}$-hran} a mód \textbf{Invariantních hran}.

\begin{figure}[ht]
    \centering
    \includegraphics[scale = 0.35]{vizualizer3.png}
\end{figure}


\section{Time-line \& Milestones}
\begin{tabular}{ |p{3cm}|p{5cm}|p{3cm}| }
\hline
Datum & Milník & Způsob prezentace\\
\hline
\hline
30. dubna & Specifikace & Ukázka specifikace\\
\hline
17. května & Základní Half-Edge struktura pro práci s grafy & Ukázka kódu\\
\hline
15. červen & Testing základní \textbf{Half-Edge} struktury pro práci s grafy & Ukázka kódu\\
\hline
30. června & Celé generování & Ukázka kódu\\
\hline
31. července & Otestované generování & Ukázka kódu\\
\hline
31. srpna & Vizualizace & Ukázka programu\\
\hline
30. září & Hotový program & Ukázka programu\\
\hline
%MB: Cely plan rozepsat az do zari, pocitat s tim, ze funkcni verze by mela byt hotova do konce semestru. Rozdelit do zhruba do mesicu na zhruba 5 casti.

\end{tabular}

%\section*{Dodatek A: Vymezení pojmů} %rozepisovat průběžně, jinak vyhodit

%Nakreslení je {\it monotonní} pokud všechny jeho hrany vedou zprava doleva.

%Nakresení je  {\it simple} pokud se žádné dvě hrány nekříží více než jednou, přičemž průnik ve vrcholu se počítá také.

%Mějme $D$ simple monotonní nakreslení grafu. Poté se podívejme na orientovaný oblouk $\gamma$ reprezentující hranu $\{u, v\}$ pro různé $u$ a $v$. Pro libovolný jiný bod $w$ řekneme, že je nalevo (napravo) od $\gamma$, jestli topologický trojúhelník $uvw$ je orientovaný proti(po) směru hodinových ručiček.

%Hrana je {\it k-hrana} pokud má právě $k$ vrcholů napravo nebo nalevo $k$.
%$E_k(D)$ je počet $k$ hran v daném nakreslení $D$.
%$E_{\leq k}(D($ je počet $\leq k$ hran, kde $\leq k$ hrana je $0$ až $k$ hrana. Jejich počet je tedy $\sum_{i = 0}^{k}E_i$. Obdobně můžeme zadefinovat $E_{\leq\leq k}(D) = \sum_{i = 0}^{k}E_{leq i} = \sum_{i = 0}^{k}(k-i+1)E_i$ a zcela analogicky $E_{\leq\leq\leq k}(D)$.

\end{document}